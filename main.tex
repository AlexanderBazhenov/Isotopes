\documentclass[a5paper,openany]{book}

\usepackage{cmap}  
\usepackage[utf8]{inputenc}
\usepackage[T2A]{fontenc} 
\usepackage[russian]{babel} 
\usepackage{amsmath,amssymb} 
\usepackage{euscript,upref}  
\usepackage{array,longtable}
\usepackage{indentfirst} 
\usepackage{graphicx} 
\usepackage{stmaryrd} 
\usepackage[justification=centering]{caption}
\usepackage{calrsfs} 
\usepackage{url}
%\usepackage{index}
\usepackage{imakeidx} 
\usepackage{multirow,makecell,array}
%\usepackage{setspace} 
%\usepackage{calligra}
\usepackage{pgf,tikz}
\usepackage{pgfplots}
\usepackage{pgfplotstable}
\usepackage{subcaption}
\usepackage{ifthen}
\usepackage{subfiles}
%\usepackage{hyperref}

\usetikzlibrary{arrows.meta}
%\pgfplotsset{compat=1.7}
\pgfplotsset{compat=newest}

%%%%%%%%%%%%%%%%%%%%%%%%%%%%%%%%%%%%%%%%%%%%%%%%%%%%%%%%%%%%%%%%%%%%%%%%%%%%%%%%%%%%%%%%  
% URL проекта - https://ru.overleaf.com/project/5e954c887ac0ac0001d54ece   
%%%%%%%%%%%%%%%%%%%%%%%%%%%%%%%%%%%%%%%%%%%%%%%%%%%%%%%%%%%%%%%%%%%%%%%%%%%%%%%%%%%%%%%%
%234567890123456789012345678901234567890123456789012345678901234567890123456789012345678
%%%%%%%%%%%%%%%%%%%%%%%%%%%%%%%%%%%%%%%%%%%%%%%%%%%%%%%%%%%%%%%%%%%%%%%%%%%%%%%%%%%%%%%%

\textwidth=114truemm
\textheight=165truemm
\oddsidemargin=-1cm
\evensidemargin=\oddsidemargin
\topmargin=-1cm
\sloppy

\pagestyle{plain}
%\mathsurround=1pt
%\tolerance=400
%\hfuzz=2pt
\makeindex

\captionsetup{font=small,labelsep=period,margin=7mm} 

%%%%%%%%%%%%%%%%%%%%%%%%%%%%%%%%%%%%%%%%%%%%%%%%%%%%%%%%%%%%%%%%%%%%%%%%%%%%%%%%%%%%%%%%
% Закомментировать следующую строку, если требуется откомпилировать печатный,
% а не электронный вариант книги (в электронном варианте в библиографии добавляются DOI)
\newcommand{\electronicbook}{}%

%%%%%%%%%%%%%%%%%%%%%%%%%%%%%%%%%%%%%%%%%%%%%%%%%%%%%%%%%%%%%%%%%%%%%%%%%%%%%%%%%%%%%%%%
%
%           Определения новых команд и макросов
%    
%\DeclareMathAlphabet{\mathcalligra}{T1}{calligra}{m}{n}
%\DeclareFontShape{T1}{calligra}{m}{n}{<->s*[1.8]callig15}{}
\newcommand{\mbf}[1]{\protect\text{\boldmath$#1$}}
\newcommand{\mbb}{\mathbb}
\newcommand{\mrm}{\mathrm}
\newcommand{\mcl}{\mathcal}
\newcommand{\msf}{\mathsf}
\newcommand{\eus}{\EuScript}
\newcommand{\ov}{\overline}
\newcommand{\un}{\underline}
\newcommand{\m}{\mathrm{mid}\;}
\newcommand{\w}{\mathrm{wid}\;} 
\newcommand{\Uni}{\mathrm{Uni}\,} 
\newcommand{\Tol}{\mathrm{Tol}\,} 
\newcommand{\Uss}{\mathrm{Uss}\,} 
\newcommand{\Ab}{(\mbf{A}, \mbf{b})}
\newcommand{\Arg}{\mathrm{Arg}\;} 
\newcommand{\sgn}{\mathrm{sgn}\;} 
\newcommand{\ran}{\mathrm{ran}\,} 
\newcommand{\rer}{\mathrm{rer}\:} 
\newcommand{\pro}{\mathrm{pro}\,} 
\newcommand{\dom}{\mathrm{dom}\,} 
\newcommand{\SEV}{\mathrm{SEV}\,} 
\newcommand{\WEV}{\mathrm{WEV}\,} 
\newcommand{\Rsv}{\mathrm{Rsv}\,} 
\newcommand{\calX}{\mathrsfs{X}} 
\newcommand{\cond}{\mathrm{cond}} 
\newcommand{\mode}{\mathrm{mode}\,} 
\newcommand{\dual}{\mathrm{dual}\,} 
\newcommand{\dist}{\mathrm{dist}\,} 
\newcommand{\Dist}{\mathrm{Dist}\,} 
\newcommand{\const}{\mathrm{const}} 
\newcommand{\USS}{\varXi_\mathit{\hspace{-0.5pt}uni}} 
\newcommand{\TSS}{\varXi_\mathit{\hspace{-0.5pt}tol}} 
\newcommand{\NExt}{_{\scalebox{0.57}{$\natural$}}} 
%\newcommand{\ih}{\scalebox{0.67}[0.87]{$\Box$\hspace*{1pt}}} 
\newcommand{\ih}{\scalebox{0.7}[1.0]{$\oblong$}} 
\newcommand{\doi}[1]{
	\ifdefined\electronicbook
	%DOI:\href{http://doi.org/#1}{#1}
	DOI:#1
	\fi}%

\renewcommand{\r}{\mathrm{rad}\;} 
\renewcommand{\vert}{\mathrm{vert}\,} 

%%%%%%%%%%%%%%%%%%%%%%%%%%%%%%%%%%%%%%%%%%%%%%%%%%%%%%%%%%%%%%%%%%%%%%%%%%%%%%%%%%%%%%%%

\renewcommand{\textfraction}{0}
\renewcommand{\topfraction}{1}
\renewcommand{\bottomfraction}{1} 
\renewcommand{\indexname}{Предметный указатель}

%%%%%%%%%%%%%%%%%%%%%%%%%%%%%%%%%%%%%%%%%%%%%%%%%%%%%%%%%%%%%%%%%%%%%%%%%%%%%%%%%%%%%%%%
%
%           Определение счётчиков 
%  
\newcounter{DefNum}[section]
\newcounter{ExmpNum}[section]
\renewcommand{\theExmpNum}{\thesection.\arabic{ExmpNum}}
\newcounter{IncluDefi}
\newcounter{IreneExmp} 
\newcounter{BazhenovExmp} 
\newcounter{RadarExmp} 
% \newcounter{ConstExmp} 
% \newcounter{VarExmp} 

%%%%%%%%%%%%%%%%%%%%%%%%%%%%%%%%%%%%%%%%%%%%%%%%%%%%%%%%%%%%%%%%%%%%%%%%%%%%%%%%%%%%%%%%
%
%           Определение необходимых окружений          
%
\newtheorem{definition}{Определение}[section] 
% \newenvironment{example}% 
%   {\par\addvspace{\medskipamount}\addtocounter{ExmpNum}{1} 
	%   \noindent\textbf{Пример {\thesection}.\arabic{ExmpNum}.}}% 
%   {\hfill$\blacksquare$\par\medskip} 
\newenvironment{example}% 
{\refstepcounter{ExmpNum}%
	\par\addvspace{\medskipamount} 
	\noindent\textbf{Пример {\theExmpNum}.}
}% 
{\hfill$\blacksquare$\par\medskip} 

%%%%%%%%%%%%%%%%%%%%%%%%%%%%%%%%%%%%%%%%%%%%%%%%%%%%%%%%%%%%%%%%%%%%%%%%%%%%%%%%
%
%                Определения новых цветов
%
\definecolor{MyRed}{rgb}{0.6,0.3,0.1}
\definecolor{MyGreen}{rgb}{0.2,0.6,0.3}
\definecolor{MyBlue}{rgb}{0.3,0.5,0.85}
\definecolor{Blau}{rgb}{0.5,0.5,0.9}
\definecolor{Gray1}{rgb}{0.6,0.6,0.65}
\definecolor{Gray2}{rgb}{0.5,0.55,0.5}
\definecolor{Gray3}{rgb}{0.6,0.55,0.55}

%%%%%%%%%%%%%%%%%%%%%%%%%%%%%%%%%%%%%%%%%%%%%%%%%%%%%%%%%%%%%%%%%%%%%%%%%%%%%%%%%%%%%%%%  
%For contents 
%\renewcommand{\l@section}{\@dottedtocline{1}{0.5em}{1.5em}}
%\renewcommand{\l@subsection}{\@dottedtocline{1}{2.5em}{2.0em}}
%\makeatother
%\setlength{\marginparwidth}{2cm}

%%%%%%%%%%%%%%%%%%%%%%%%%%%%%%%%%%%%%%%%%%%%%%%%%%%%%%%%%%%%%%%%%%%%%%%%%%%%%%%%%%%%%%%%

\title{Интервальная таблица Менделеева\\* 
	и изотопы на Земле}

\author{А.Н.\,Баженов, А.Ю.\,Тельнова}

%%%%%%%%%%%%%%%%%%%%%%%%%%%%%%%%%%%%%%%%%%%%%%%%%%%%%%%%%%%%%%%%%%%%%%%%%%%%%%%%%%%%%%%%

\begin{document}
	
	\maketitle 
	\newpage
	УДК 519.9
	Р32\\
	
	Аннотация 
	
	
	В книге рассмотрена современная версия таблицы Менделеева с интервальными значениями атомных весов, разработанная Международным союзом теоретической и прикладной химии IUPAC. 
	
	Рассмотрение проведено с разных позиций. Исходно таблица Менделеева носила <<химический>> характер. Она описывала свойства элементов с точки зрения их способности формировать молекулы --- составляющие природные и синтезированные человеком соединения. Классификация элементов по группам и периодам сообразно атомным массам позволила правильно расположить известные элементы в периодическом порядке, уточнить атомные веса, предсказать существование новых ранее неизвыестных элементов.
	Со временем были юоавлены ранее отсутсвующие элементы и целые группы: благородные газы, редкоземельные элементы и актиноиды.
	
	С отркрытием радиоактивности началось развитие ядерной физики, что привело к созданию модели атома и атомного ядра. Атомные массы были заменены на заряды ядра. Было выяснено, что атомные массы элементов не есть их базовая характеристика. Напротив, то что казалось фундаментом материи, оказалось архипелагом островков стабильных изотопов в море возможных ядерных конфигураций.  Изотопные композиции оказались преимущественным способом существования элементов в природе, а атомные веса --- производными величинами.
	
	С выходом в Космос и развитием инструментальных методов, их примеения в науках о Земле и биологии, созданием планетной и звездной космохимии, изотопной планетологии и изучения элементного и изотопного состава звёзд, ситуация с атомными весами в таблице Менделеева ещё раз изменилась.  
	
	Теперь следует говорить о  таблице Менделеева как способе описания элементов на планете Солнечной системы Земле с ее специфичной геологической историей, с взаимодействием с Космосом в конкретном месте Галактики и в текущее время планетарной эволюции.
	На других планетах и иных космических объектах, на других этапах их геологического развития и существования специфичиеских форм жизни, наличие и отсутствие тех или иных элементов и их атомные массы другие.
	
	Такое положение вещей необходимо осозанвать, изучать и применять. Для первого знакомства с предметов в книге   
	приводятся необходимые сведения по истории вопроса, ядерной физике и нуклеосинтезу, об изотопных распределениях в живой и неживой природе на Земле.
	
	Для математического описания и практических вычислений представлены понятия, методы и инструменты анализа данных с интервальной неопределённостью применительно к тематике изотопов.
	
	Книга адресована всем, кто интересуется современным естествознанием в различных областях и применению математики к решению практических задач.
	
	
	
	
	%%%%%%%%%%%%%%%%%%%%%%%%%%%%%%%%%%%%%%%%%%%%%%%%%%%%%%%%%%%%%%%%%%%%%%%%%%%%%%%%%%%%%%%%
	
	\tableofcontents      %  Содержание  
	
	%%%%%%%%%%%%%%%%%%%%%%%%%%%%%%%%%%%%%%%%%%%%%%%%%%%%%%%%%%%%%%%%%%%%%%%%%%%%%%%%%%%%%%%%
	%%%%%%%%%%%%%%%%%%%%%%%%%%%%%%%%%%%%%%%%%%%%%%%%%%%%%%%%%%%%%%%%%%%%%%%%%%%%%%%%%%%%%%%%
	
	\chapter*{Введение}
	\addcontentsline{toc}{chapter}{Введение}   
	
	
	В середине 19 века многие исследователи пытались найти закономерности химических свойств элементов. Наиболее удачной оказалась система Д.И.Менделеева. Статья <<Соотношение свойств с атомным весом элементов>> была опубликована в 1869 г. \cite{Mendeleev1869ru}. В то же время таблица была аннотирована в Германии \cite{Mendeleev1869} и стала достоянием научной общественности. В дальнейшем Менделеев развил свои идеи и дал название предложенной системе  \emph{Периодический закон}. \index{Менделеев} \index{Периодический закон}
	
	\ldots
	
	
	Истории создания Периодической таблицы элементов посвящена обширная библиография. В книгах \cite{Trifonov1974, Scerri2019} дано краткое изложение основных вех этого процесса, а также освещены экспериментальные исследования и создание теории электронного строения атомов и ядер химических элементов, которые привели к созданию науки об изотопах. \index{Периодическая таблица элементов} В настоящее время идёт процесс более подробного представления этой системы, которая теперь называется \emph{Периодическая таблица элементов и изотопов}.
	\index{Периодическая таблица элементов и изотопов} Тем самым привычная таблица Менделеева дополняется ещё одним измерением --- ядерным. \index{изотопы} 
	
	\ldots
	
	В книге четыре главы. 
	
	В первой главе даются исторические сведения о развитии учения о периодичности, атомных весах, ядерной физике. 
	
	Следующие две главы носят описательный характер, используются опубликованные в периодической научной печати материалы. 
	
	Во второй главе приводятся сведения по ядерной физике и нуклеосинтезу, об изотопных распределениях в живой и неживой природе на Земле.
	В третьей главе
	рассмотрена современная версия таблицы Менделеева с интервальными значениями атомных весов, разработанная Международным союзом теоретической и прикладной химии IUPAC. 
	
	В четвертой главе
	рассмотрены понятия, методы и инструменты анализа данных с интервальной неопределённостью применительно к тематике изотопов.
	Кроме этого, обсуждается возможность применения нечётких множеств.
	Изложение базового материала опирается на сложившиеся математические понятия и методы.
	Их конкретное применение к изотопной тематике только начинается. 
	
	\chapter{Атомные веса, возникновение ядерной физики, \ldots}
	
	\paragraph{Развитие учения о периодичности.}
	Приведём некоторые исторические вехи развития \cite{Trifonov1974, Scerri2019, Bekman} \ldots Переход от исследования веществ и эмпирического исследования атомов элементов к исследованию строения атомов и элементарных частиц
	
	{\small	
		%{\tiny 
			\begin{tabular}{l|l|l}
				Год & Автор	& Проблема \\
				\hline
				1885 & Ридберг & {\color{red}Атомные веса не могут рассматриваться} \\
				~ & ~ & {\color{red} в качестве независимой переменной} \\	
				~ & Балмер & Формула для спектральных линий водорода \\	
				1886 & Крукс & {\color{red}Атомные веса не одинаковы для всех атомов } \\
				1888 & ~ &   {\color{red}элемента, а существует распределение }\\
				1890 & Ридберг & Обобщение формулы Бальмера на разные элементы \\	
				1895 & Рентген & Открытие X-лучей \\
				~ & Стрэтт, Рамзай & Аргон --- новая составляющая часть атмосферы \\
				1896 & Беккерель & Радиоактивность урановых соединений  \\
				~ & ~ &  и металлического урана\\
				1897 & Томсон & Катодные лучи --- носитель отрицательного заряда  \\
				~ & ~ &  для всех веществ, в 1800 раз легче водорода\\	
				~ & Ридберг & {\color{red}Атомный вес элементов $M = N+D$,} \\	
				~ & ~ &  {\color{red}$N$ --- целое, $D$ --- малая периодическая функция}\\			
				\hline
			\end{tabular}
		}
		
		\paragraph{Исследования атома и ядра.} По мере осознания закономерностей строения атома и накопления эмпирического материала \ldots
		
		{\small	
			%{\tiny 
				\begin{tabular}{l|l|l}
					Год & Автор	& Открытие \\
					\hline
					1913 & Дж.Дж.Томсон & Открытие {\color{red}изотопов} неона с массой 20 и 22\\
					~ & А.Ван-ден-Брук & {\color{red}Порядковый номер элемента в Периодической}  \\
					~ & ~ &  {\color{red}системе равен заряду ядра его 	атомов} \\
					~ & Ф.Содди & Понятие изотопа у радиоактивных
					элементов   \\
					1914, & Н.Бор & Свойства элементов периодической системы  \\
					1921 & ~ & суть  функции зарядов ядер их атомов \\
					1914 & Г.Мозли & Зависимость  частоты характеристического	  \\
					~ & ~ & излучения от порядкового номера элементов \\
					1916 & У.Д.Харкинс & {\color{red}Правило большей распространенности элементов}\\
					~ & ~ & {\color{red}с четными порядковыми номерами} \\
					1917 & Ф.Содди  & Изотопы высшего порядка --- ядерные изомеры\\
					1918 & Дж.Дж.Томсон  &  Доказательство существования изотопов среди\\
					~ & ~ & продуктов радиоактивного распада \\
					1918 & А.У.Стюарт  &  Открытие {\color{red}изобар}\\
					1919 &Э.Резерфорд &   Открытие протона; доказательство  \\
					~ & ~ & наличия в ядрах элементов протонов; \\
					~ & ~ & первая искусственная ядерная реакция \\
					~ & ~ & --- превращение азота в кислород \\
					1920 & Г.Хевеши & Явление изотопного обмена\\
					1921 & Н.Бор & Строение атомов --- связь периодичности  \\
					~ & ~ & их химических и спектральных свойств с \\
					~ & ~ &  характером формирования электронных конфигураций \\
					~ & ~ &  по мере роста заряда ядра \\
					1921 & О.Ган  & Открытие изомера урана\\
					1921 & Ф.Астон  & {\color{red}212 природных изотопов различных элементов;} \\
					~ & ~ &   {\color{red}Массы изотопов --- целые числа} \\
					\hline
				\end{tabular}
			}	
			
			
			\chapter{Изотопы элементов в природе}
			
			
			
			В настоящей вводной главе даётся информация, необходимая для понимания важности изотопов:
			по ядерной физике и нуклеосинтезу, об изотопных распределениях в живой и неживой природе на Земле.
			
			
			\section{Сведения из ядерной физики}
			
			\subsection{Экспериментальные факты про ядра, нуклоны, \ldots}
			
			\subsection{Модели ядра}
			{NZ-диаграммы}
			\subsection{Изобары}
			
			
			\index{изотопы}
			\index{изобары}
			\index{NZ-диаграммы}
			
			
			\section{Ядерный нуклеосинтез}
			
			\section{Изотопы на Земле}
			
			Фракционирование изотопов в природе
			
			\subsection{Геология}
			\subsection{Биология}
			
			
			
			
			\chapter{Периодическая таблица элементов и изотопов} 
			
			В главе представлена современная версия таблицы Менделеева с интервальными значениями атомных весов, разработанная Международным союзом теоретической и прикладной химии IUPAC. \index{IUPAC, Международный союз теоретической и прикладной химии}
			
			\chapter{Использование  данных с интервальной неопределенностью}
			
			Заключительная глава книги, в отличие от предыдущих, содержит оригинальный материал.
			На настоящее время способы работы с изотопными данными содержатся только в различных рабочих документах IUPAC.  Авторы публикаций не являются специалистами по интервальному анализу, и в них содержатся не вполне верные утверждения.
			
			Мы рассмотрим различные способы представления изотопных данных и работы с ними.
			
			Для описания данных с интервальной неопределенностью возможны различные варианты.          
			\begin{itemize}
				\item интервалы
				\item мультиинтервалы
				\item твины
				\item нечёткие множества
				\item функции распределения
			\end{itemize}
			Первые три варианта относятся к классической интервальной арифметике и её расширениям. При таком способе описания базовым понятием является \emph{интервал} \index{интервал} --- бесструктурный элемент вещественной оси, отрезок. Нечёткие множества представляют способ описания множества, имеющим структуру.  Наконец, функции распределения типичны для теоретико-вероятностной статистистики и во многих случаях имеют аналитическое выражение. \index{мультиинтервал}     \index{твин}     \index{нечёткое множество}
			
			\section{Анализ данных с интервальной неопределенностью}
			
			В этом разделе представлена краткая информация о классической интервальной арифметике и её расширениях, которые можно использовать для  описания изотопных данных: полная интервальная арифметика Каухера, объединения интервалов --- мультиинтервалы, интервалы с интервальными вершинами --- твины. \index{классическая интервальная арифметика} \index{полная интервальная арифметика Каухера}
			
			\ldots
			
			
			\section{Анализ данных с использованием нечётких множеств}
			
			Приведём базовые сведения о нечётких методах описания \cite{BookInteStat}.
			
			При нечётком описании результатов измерений и наблюдений мы полагаем, что вместо их 
			точных значений нам известны так называемые функции принадлежности нечётких чисел, 
			возникающих в результате измерений. 
			\index{нечёткие методы} 
			
			\emph{Нечётким множеством} называется множество $X$, 
			образованное элементами произвольной природы, которое дополнено так называемой 
			\emph{функцией принадлежности} $\mu: X\to[0, 1]$, значение которой $\mu(x)$ на элементе 
			$x\in X$ показывает <<степень принадлежности>> $x$ множеству $X$. 
			У стандартной функции принадлежности множества (называемой также \emph{индикаторной 
				функцией} множества) значения могут быть равны только $0$ или $1$: $0$ соответствует 
			состоянию <<не является элементом множества>>, а $1$ соответствует состоянию <<принадлежит
			множеству>>. Допущение для функции $\mu$ непрерывного ряда значений из интервала $[0, 1]$ 
			позволяет характеризовать ситуации с <<частичной принадлежностью>> множеству $X$, когда 
			мы не вполне уверены, принадлежит ли элемент множеству. В новой ситуации мы можем 
			оперировать количественной мерой этой принадлежности и строить на её основе наши 
			выводы и заключения. \index{нечёткое множество}\index{функция принадлежности} 
			
			%%%%%%%%%%%%%%%%%%%%%%%%%%%%%%%%%%%%%%%%%%%%%%%%%%%%%%%%%%%%%%%%%%%%%%%%%%%%%%%%%%%%%%%%%%%%%%%%
			\addcontentsline{toc}{chapter}{Литература} 
			\begin{thebibliography}{00}
				\bibitem{Mendeleev1869ru}
				\textsc{Менделеев, Д.} (1869). “Соотношение свойств с атомным весом элементов”. Журнал Русского Химического Общества. 1: 60—77.
				
				\bibitem{Mendeleev1869}
				\textsc{Mendeleev, Dmitri} (1869). “Versuche eines Systems der Elemente nach ihren Atomgewichten und chemischen Functionen”. Journal für Praktische Chemie. 106: 251.
				
				\bibitem{Trifonov1974}
				\textsc{Трифонов Д.Н., Кривомазов А.Н., Лисневский Ю.И.} Учение о периодичности и учение о радиоактивности (комментированная хронология важнейших событий). М., Атомиздат, 1974, 248 с.
				
				\bibitem{Scerri2019}
				\textsc{E. Scerri}
				The Periodic Table. Its Story and Its Significance. 
				2nd edition.  New York, NY : Oxford University Press, 2019 
				
				
				\bibitem{Bekman}
				Бекман, И. Н.  Атомная и ядерная физика: радиоактивность и ионизирующие излучения : учебник для вузов / И. Н. Бекман. — 2-е изд., испр. и доп. — Москва : Издательство Юрайт, 2022. — 493 с. — (Высшее образование). — ISBN 978-5-534-08692-8. 
				
				
				\bibitem{BookInteStat} \textsc{А.Н.\,Баженов, С.И.\,Жилин, С.И.\,Кумков, С.П.\,Шарый.} <<Обработка и анализ данных с интервальной неопределённостью>>. 
				
			\end{thebibliography}
			%%%%%%%%%%%%%%%%%%%%%%%%%%%%%%%%%%%%%%%%%%%%%%%%%%%%%%%%%%%%%%%%%%%%%%%%%%%%%%%%%%%%%%%%%%%%%%%%
			
			\addcontentsline{toc}{chapter}{Предметный указатель} 
			\raggedright\small\printindex 
			
		\end{document}
